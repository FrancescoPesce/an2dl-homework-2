\documentclass[11pt]{article}
\usepackage[utf8]{inputenc}
\usepackage[english]{babel}
\usepackage{amsmath}
\usepackage{graphicx}
\usepackage{float}
\usepackage{lipsum}
\usepackage{multicol}
\usepackage{xcolor}
\usepackage{tabularx}
\usepackage{booktabs}
\usepackage{hyperref}
\newcolumntype{Y}{>{\centering\arraybackslash}X}
\usepackage[left=2.00cm, right=2.00cm, top=2.00cm, bottom=2.00cm]{geometry}
\usepackage{enumitem}

\title{AN2DL Reports Template}

\begin{document}

\begin{figure}[H]
      \raggedright
      \includegraphics[scale=0.4]{polimi.png} \hfill
      \includegraphics[scale=0.3]{airlab.jpeg}
\end{figure}

\vspace{5mm}

\begin{center}
      % Select between First and Second
      {\Large \textbf{AN2DL - Second Homework Report}}\\
      \vspace{2mm}
      % Change with your Team Name
      {\Large \textbf{LosPollosHermanos}}\\
      \vspace{2mm}
      % Team Members Information
      {\large Mohammadhossein Allahakbari,}
      {\large Michele Miotti,}
      {\large Francesco Pesce}\\
      \vspace{2mm}
      % Codabench Nicknames
      {mh2033,}
      {michelem,}
      {francescopesce}\\
      \vspace{2mm}
      % Matriculation Numbers
      {246639,}
      {249499,}
      {247974}\\
      \vspace{5mm}
      \today
\end{center}
\vspace{5mm}

\begin{multicols}{2}
      % Note: The following sections represent a suggested
      % structure. We don't need to follow it strictly.

      % -----------------------------------------------------------------------
      % INTRODUCTION
      % -----------------------------------------------------------------------
      \section{Introduction}
      % In this section, you should present your project's context and
      % objectives. You might want to:
      % \begin{itemize}
      %     \item Dehe problem (\textit{you may use italics to highlight
      %               definitions})
      %     \item State your goals (\textbf{emphasise key points with bold})
      %     \item Outline your approach
      % \end{itemize}

      % \noindent For instance, you might write: ``This project focuses on
      % \textit{image classification} using \textbf{deep learning} techniques."

      This report presents the results of the \textit{semantic segmentation}
      task proposed in the second homework of the \texttt{Artificial Neural Networks and Deep Learning} course. The goal of the project is to categorize all the pixels in a 64x128 grayscale image into different classes, each representing a different type of terrain.

      % -----------------------------------------------------------------------
      % PROBLEM ANALYSIS
      % -----------------------------------------------------------------------
      \section{Problem Analysis}
      % Here you can discuss your initial analysis of the problem. Consider
      % including:
      % \begin{enumerate}
      %     % 8 classes, 96x96 rgb images, labels, etc
      %     \item Dataset characteristics
      %     \item Main challenges % The test set is horrible
      %     That the test set was not horrible? Is this what they mean?
      %     \item Initial assumptions 
      % \end{enumerate}

      % \noindent If you need to reference papers, use the citation command:
      % Recent work~\cite{lecun2015deep} suggests..."

      After manually analyzing the dataset, we found that a portion of the test presented irrelevant, or otherwise unnecessary images. We didn't consider said images in our training, as we didn't expect them to be present in the test set.

      % -----------------------------------------------------------------------   
      % METHOD
      % -----------------------------------------------------------------------
      \section{Method}
      % Not sure what they are asking here. The final model?
      % This section should detail your approach. You can use equations to
      % explain your methodology. For example, a simple model representation:
      % \begin{equation}
      %     \label{eq:model}
      %     f(x) = \text{softmax}(Wx + b)
      % \end{equation}

      % \noindent Or a more complex loss function:
      % \begin{equation}
      %     \label{eq:loss}
      %     \mathcal{L} = -\frac{1}{N}\sum_{i=1}^{N} y_i\log(\hat{y}_i)
      % \end{equation}

      % \noindent Reference these equations in your text, like:``As shown in
      % equation~\ref{eq:model}..."

      Most of the development process was done on a \texttt{Jupyter} notebook, using the \texttt{Tensorflow}\cite{TensorFlow} and \texttt{Keras}\cite{chollet2015keras} libraries for \texttt{Python}, mostly run locally using an \texttt{NVIDIA RTX 2060 Mobile} for training, due to strict GPU usage limits of popular platforms, such as \texttt{Google Colab}.

      % -----------------------------------------------------------------------
      % EXPERIMENTS
      % -----------------------------------------------------------------------
      \label{sec:experiments}
      \section{Experiments}
      % Step 1: bad CNN just to test things out
      % Step 2: normal augmentation
      % Step 3: overlaying
      % Step 4: keras cv (gigachad picture)
      % Step 5: testing residual networks (up to 0.8)
      %   and inception blocks (up to 0.72)
      % Step 6: custom block (in detail)
      % Step 7: add more filters
      % Step 8.....: not done yet
      % If you need accuracies, number of parameters etc ask me

      % Failed experiments:
      % transfer learning (bad accuracy) (VGG, MobileNet, ResNet), both with and
      % without fine tuning
      % Lion, stochastic gradient descent with momentum

      % Ideas that worked momentarily:
      % voting mechanism with multiple neural networks
      % averaging weights below a certain validation loss
      We performed many experiments, some of which significantly improved the accuracy of our models. In chronological order, the main breakthroughs were:
      \begin{itemize}[leftmargin=*]
            \setlength\itemsep{0em}
            \item \textbf{Experiment:} Description
      \end{itemize}

      % For your experiments, you might want to present your results in tables.
      % Here's an example of a wide table comparing different models:

      % \begin{table*}[t]
      %     \centering
      %     \setlength{\tabcolsep}{3pt}
      %     \caption{An example of wide table. Best results are highlighted in
      %         \textbf{bold}.}
      %     \begin{tabularx}{\textwidth}{lYYYc}
      %         \toprule
      %         Model            & Accuracy                  & Precision
      %                          & Recall                    & ROC AUC
      %         \\
      %         \midrule
      %         VGG18            & 72.20 $\pm$ 3.06          & 94.95 $\pm$ 0.52
      %                          &
      %         86.95 $\pm$ 0.55 & 80.16 $\pm$ 0.81
      %         \\
      %         Custom Model     & 27.71 $\pm$ 3.19          & 75.70 $\pm$ 1.07
      %                          & 55.75 $\pm$ 2.16          & 36.60 $\pm$ 1.26
      %         \\
      %         ResNet18         & \textbf{89.24 $\pm$ 2.38} & \textbf{95.54
      %         $\pm$ 0.49}      & \textbf{93.43 $\pm$ 1.30} & \textbf{91.68 $\pm$
      %             0.71}
      %         \\
      %         \bottomrule
      %     \end{tabularx}
      %     \label{tab:Performance}
      % \end{table*}

      % \noindent For more specific measurements, you might use a narrower
      % table:

      % \begin{table}[H]
      %     \centering
      %     \setlength{\tabcolsep}{3pt}
      %     \caption{An example of table. Best results may be highlighted in
      %         \textbf{bold}.}
      %     \begin{tabularx}{\linewidth}{lY}
      %         \toprule
      %         Time [$\mu$s] & Distance [mm] \\
      %         \midrule
      %         22$\pm$4      & 8$\pm$1       \\
      %         17$\pm$3      & 7$\pm$1       \\
      %         15$\pm$3      & 6$\pm$1       \\
      %         13$\pm$2      & 5$\pm$1       \\
      %         10$\pm$2      & 4$\pm$1       \\
      %         8$\pm$2       & 3$\pm$1       \\
      %         5$\pm$1       & 2$\pm$1       \\
      %         37$\pm$1      & 1$\pm$1       \\
      %         \bottomrule
      %     \end{tabularx}
      %     \label{tb:Measurements}
      % \end{table}

      % \noindent You can also include figures to visualise your results:
      % \begin{figure}[H]
      %     \centering
      %     \includegraphics[width=0.75\linewidth]{random.jpeg}
      %     \caption{Example figure showing [describe what the figure shows]}
      %     \label{fig:results}
      % \end{figure}

      % \noindent Reference figures using like:``As shown in
      % Figure~\ref{fig:results}..."

      % -----------------------------------------------------------------------
      % RESULTS
      % -----------------------------------------------------------------------
      \label{sec:results}
      \section{Results}
      % Keras CV is good
      % Combining residual and inception is good
      % Present your main findings here. You might want to:
      % \begin{itemize}
      %     \item Compare your results with baselines
      %     \item Highlight key achievements using \textbf{bold text}
      %     \item Explain any unexpected outcomes
      % \end{itemize}

      % -----------------------------------------------------------------------
      % DISCUSSION
      % -----------------------------------------------------------------------
      \section{Discussion}
      % Wait until we have the final results
      % In this section, analyse your results critically. Consider:
      % \begin{itemize}
      %     \item Strengths and weaknesses
      %     \item Limitations and assumptions
      % \end{itemize}

      % -----------------------------------------------------------------------
      % CONTRIBUTIONS
      % -----------------------------------------------------------------------
      \section{Contributions}

      % -----------------------------------------------------------------------
      % CONCLUSIONS
      % -----------------------------------------------------------------------
      \section{Conclusions}
      % Summarise your work and discuss potential future directions. This is
      % where you can:
      % \begin{itemize}
      %     \item Restate main contributions
      %     \item Suggest improvements
      %     \item Propose future work
      % \end{itemize}

      % Remember to include the bibliography!
      \bibliography{references}
      \bibliographystyle{abbrv}

\end{multicols}
\end{document}